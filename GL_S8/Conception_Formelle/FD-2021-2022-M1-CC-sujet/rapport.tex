\documentclass[a4paper]{book}
\usepackage{fullpage}

\usepackage[utf8]{inputenc}
\usepackage[T1]{fontenc}
\usepackage[francais]{babel}

\usepackage{latexsym}
\usepackage{fancyhdr}
\usepackage{makeidx}
\usepackage{graphics}
\usepackage{graphicx}
\usepackage{longtable}
\usepackage{moreverb}
\usepackage{listings}

\newcommand{\altarica}{{\sc AltaRica}}

\begin{document}

\title{Master 1, Conceptions Formelles\\
Projet du module \altarica\\
Synthèse (assistée) d'un contrôleur du niveau d'une cuve}

\date{}

\author{Garion GOUBARD \and Kristo DHIMA}

\maketitle

\chapter{Le sujet}
\input{tank}

\chapter{Le rapport}
Le rapport est sur 20 points.

\section{Processus}
\subsection{Rôle du fichier {\tt GNUmakefile} (1.5 points)}
Le fichier {\tt GNUmakefile} sert à fabriquer le modèle {\tt tank}. Il fabrique tous les
différents controleurs, par exemple avec une défaillance, avec deux défaillances ou avec
une vanne virtuelle par exemple.

\subsection{Rôle de la constante {\tt nbFailures} et de l'assertion associée (0.5 point)}
La constante nbFailures est utilisé dans le fichier {\tt GNUmakefile} pour définir le
nombre de contrôleurs à fabriquer, tandis que l'assertion associée sert a borner le nombre
de vannes pouvant tomber en panne simultanement.

\section{Résultats avec le contrôleur initial {\tt Ctrl}}
\subsection{Calcul d'un contrôleur}
\subsubsection{Avec 0 défaillance (0.5 point)}
\input{LaTeX/System0FCtrl.tex}
\paragraph{Interprétation des résultats}
En ce qui concerne le système System0FCtrl, nous pouvons constater que le système est
stable et efficace. Il n'y a pas de défaillance, et il n'y a pas de blocages. Meme si dans
l'etat initial on a 86 sitations critiques, le controleur est capable de les regler sans
probleme. Ce controleur n'a donc pas besoin d'optimisation, car il propose le debit ideal
en evitant les blocages et situations critiques.

\subsubsection{Avec 1 défaillance (0.5 point)}
\input{LaTeX/System1FCtrl.tex}
\paragraph{Interprétation des résultats}
Le système System1FCtrl commence par ne pas avoir de blocages, mais des situations critiques.
Apres une iteration, il est capable de diminuer le nombre de situations critiques, mais en
contrepartie il en cree des blocages. Apres deux et trois iterations, le resultat devient
un peu mauvais. 3 situations de blocage sont apparues. Il est important aussi de noter que
tous les cas redoutés sont des blocages, dont environ 2/3 sont des situations critiques.
Des optimisations sont nécessaires afin d'éviter ce grand nombre de blocages et de situations
critiques.


\subsubsection{Avec 2 défaillances (0.5 point)}
\input{LaTeX/System2FCtrl.tex}
\paragraph{Interprétation des résultats}
Pareillement au système System1FCtrl, le système System2FCtrl commence par ne pas avoir de
blocages, mais des situations critiques. Après une iteration, il est capable de diminuer le
nombre de situations critiques, mais en contrepartie il créé des blocages. Apres plusieurs
iterations, le résultat devient un peu mauvais. Environ 40 situations de blocage sont apparues.
Il est aussi important de noter que tous les cas redoutés sont des blocages, dont environ
1/3 sont des situations critiques, comparé a 2/3 du système System1FCtrl.Des optimisations sont
nécessaires afin d'éviter ce grand nombre de blocages et de situations critiques.



\subsubsection{Avec 3 défaillances (0.5 point)}
\input{LaTeX/System3FCtrl.tex}
\paragraph{Interprétation des résultats}
Le système System3FCtrl commence par avoir beaucoup de sitations critiques, 617 pour être
exact. Après une iteration, il est capable de neutraliser les situations critiques, mais en
contrepartie il créé un petit nombre de blocages, comparé avec les autres systèmes. Après
plusieurs iterations, le résultat devient plutôt bon. Le nombre de blocages diminue
de 112 a 27. Des petites optimiations sont nécessaires afin d'éviter ce petit nombre
de blocages, mais le système est encore assez efficace.

\subsection{Bilan avec le contrôleur initial (1 point)}
Le controleur initial permet de regler une bonne partie de situations critiques, mais il
n'est pas capable de les regler toutes. En plus, il cree en contrepartie un bon nombre
de blocages, suivant le nombre de situations critiques.

\section{Construction d'un contrôleur initial plus performant}
\subsection{Rôle du composant {\tt ValveVirtual}(2 points)}
\includegraphics[height=.2\textheight,width=.5\textwidth]{Graphs/Valve-modes.pdf}
\includegraphics[height=.2\textheight,width=.5\textwidth]{Graphs/ValveVirtual-modes.pdf}
La semantique du composant {\tt ValveVirtual} est le fait que nous pouvons pas utiliser
la valve si elle est bloquée. Elle simule donc une valve reele parfaite. Le role de ce
composant est donc d'eviter les utilisations inutiles de la valve tant qu'on ne peut pas
changer le rate avec et de simuler les defaillances.

\subsection{Rôle du composant {\tt CtrlVV} (4 points)}
Le rôle du composant {\tt CtrlVV} est de simuler le contrôleur initial {\tt Ctrl} avec une
valve virtuelle. L'utilité de ce composant est de permettre de tester le contrôleur initial
{\tt Ctrl} avec une valve virtuelle, en evitant donc de faire des coups de changement
qui menent a une erreur, par exemple un blogage ou une situation critique.


\section{Résultats avec le contrôleur {\tt CtrlVV}}
\subsection{Calcul d'un contrôleur}
\subsubsection{Avec 0 défaillance (0.5 point)}
\input{LaTeX/System0FCtrlVV.tex}
\paragraph{Interprétation des résultats}
Le système System0FCtrlVV est stable et efficace. Il commence par avoir 86
situations critiques. Au bout de la première itération, il est capable de
régler toutes ces situations critiques sans générer de blocage ni de défaillance.
Ce système est tres efficace, donc il n'y a pas besoin de l'optimiser.

\subsubsection{Avec 1 défaillance (0.5 point)}
\input{LaTeX/System1FCtrlVV.tex}
\paragraph{Interprétation des résultats}
Le système System1FCtrlVV commence par avoir beaucoup de situations critiques, 413 pour
être exact. Au bout de la première itération, il est capable de régler presque toutes ces
situations critiques en generant qu'un petit nombre de blocages. Au bout de la 3ème itération,
il est capable de neutraliser le nombre de blocages. Ce système est assez efficace, mais il
peut avoir une défaillance, donc il y a quand meme un besoin d'optimiser, afin de le rendre
un peu plus stable et rapide.


\subsubsection{Avec 2 défaillances (0.5 point)}
\input{LaTeX/System2FCtrlVV.tex}
\paragraph{Interprétation des résultats}
Le système System2FCtrlVV commence par avoir vraiment beaucoup de situations critiques, 812 pour être
exact. Au bout de la première itération, il est capable de régler presque toutes ces situations critiques
en generant quand meme un bon nombre de blocages, 70 pour etre exact. Au bout de la 2ème itération,
il est capable de neutraliser le nombre de blocages. Ce système est assez efficace, plus efficace
que le systeme System1FCtrlVV mais il presente un risque plus haut de defaillance. Le besoin
d'optimiser est donc plus important.

\subsubsection{Avec 3 défaillances (0.5 point)}
\input{LaTeX/System3FCtrlVV.tex}
\paragraph{Interprétation des résultats}
Le système System3FCtrlVV commence par avoir beaucoup de situations critiques, 970 pour être
exact. Au bout de la première itération, il est capable de régler presque toutes ces situations critiques
en génerant quand même un bon nombre de blocages, 97 blocages. Au bout de la 2ème itération, il est capable de
neutraliser le nombre de blocages. Ce système est assez efficace, presque aussi efficace que le systeme
System2FCtrlVV, mais il presente un risque plus haut de defaillance. Le besoin d'optimiser est donc encore plus
important.


\subsection{Bilan avec le contrôleur CtrlVV (1 point)}
Le contrôleur CtrlVV permet d'avoir de meilleurs résultats que le contrôleur initial
grâce a la Valve Virtuelle. Il est beaucoup plus stable et donne des résultats beaucoup plus cohérents.
Il permet de totalement neutraliser les situations critiques et les blocages. Cela est dû au fait qu'il
est capable de prevoir une situation de blocage, grace a la valve virtuelle.

\section{Une première optimisation des contrôleurs pour améliorer le débit aval}
\subsection{Une optimisation basée sur les priorités (1 point)}
\small{\lstinputlisting{ControleursOpt/Optimisation.alt}}
% A COMPLETER en expliquant en quoi consiste l'optimisation mise en place.

\subsection{Calcul des contrôleurs optimisés avec {\tt CtrlVV}}
\paragraph{Avec 0 défaillance}\ \\
\input{LaTeX/System0FCtrlVV_opt.tex}

\paragraph{Avec 1 défaillance}\ \\
\input{LaTeX/System1FCtrlVV_opt.tex}

\paragraph{Avec 2 défaillances}\ \\
\input{LaTeX/System2FCtrlVV_opt.tex}

\paragraph{Avec 3 défaillances}\ \\
\input{LaTeX/System3FCtrlVV_opt.tex}

\subsection{Bilan avec la première optimisation du contrôleur CtrlVV (1 point)}
% A COMPLETER

\section{Une deuxième optimisation (3 points)}
Il est possible d'obtenir de meilleurs résultats que les précédents par au moins deux façons.
\begin{enumerate}
\item En utilisant un meilleur ordre pour les priorités entre événements.
\item En introduisant cet objectif dans le système de calcul de point fixe des actions du contrôleur.
\end{enumerate}

Vous devez proposer une des deux optimisations conduisant à une solution dans laquelle le débit de la vanne aval est le moins souvent possible décrémenter, voire jamais. Pour cela, vous pouvez~:
\begin{itemize}
\item Meilleur ordre sur les événements.
  \begin{enumerate}
  \item Modifier le fichier \texttt{ControleursOpt/Optimisation.alt}.
  \item Faites \texttt{make}.
  \item Quand vos résultats sont satisfaisants, notez les, puis copiez votre fichier dans \texttt{ControleursOpt/Optimisation-2.alt}.
  \item Remettez le fichier \texttt{ControleursOpt/Optimisation.alt} d'origine.
  \item Faites \texttt{make}.
  \end{enumerate}
\item Meilleur système d'équations au point fixe.
  \begin{enumerate}
  \item Modifier le fichier \texttt{Spec/System.spe}.
  \item Faites \texttt{make}.
  \item Quand vos résultats sont satisfaisants, notez les, puis copiez votre fichier dans \texttt{Spec/System-2.spe}.
  \item Remettez le fichier \texttt{Spec/System.spe} d'origine.
  \item Faites \texttt{make}.
  \end{enumerate}
\end{itemize}

\paragraph{Le nouvel ordre}\ \\
\small{\lstinputlisting{ControleursOpt/Optimisation-2.alt}}
% A COMPLETER en décrivant les résultats obtenus

\paragraph{Le nouveau système d'équations}\ \\
\small{\lstinputlisting[texcl]{Spec/System-2.spe}}
% A COMPLETER en décrivant les résultats obtenus

\section{Conclusion sur la synthèse de contrôleur (1 point)}
Le contrôleur initial n'est pas tres efficace, mais il donne quand même des résultats convenables.
Il peut être amélioré grâce a la valve virtuelle que le CtrlVV implémente. On peut voir que ce deuxième
contrôleur est bien plus efficace et plus stable. Il permet de neutraliser les situations critiques et de
régler les blocages. Il y a quand même plusieurs optimisations à faire afin d'éviter des défaillances, ainsi
qu'améliorer le nombre d'itérations nécessaire pour régler les situations critiques et les blocages.

\end{document}
